\documentclass[12pt]{article}
 
\usepackage[margin=1in]{geometry}
\usepackage{amsmath,amsthm,amssymb}
\usepackage{physics}
 
\newcommand{\N}{\mathbb{N}}
\newcommand{\R}{\mathbb{R}}
\newcommand{\Z}{\mathbb{Z}}
\newcommand{\Q}{\mathbb{Q}}
 
\newenvironment{theorem}[2][Theorem]{\begin{trivlist}
\item[\hskip \labelsep {\bfseries #1}\hskip \labelsep {\bfseries #2.}]}{\end{trivlist}}
\newenvironment{lemma}[2][Lemma]{\begin{trivlist}
\item[\hskip \labelsep {\bfseries #1}\hskip \labelsep {\bfseries #2.}]}{\end{trivlist}}
\newenvironment{exercise}[2][Exercise]{\begin{trivlist}
\item[\hskip \labelsep {\bfseries #1}\hskip \labelsep {\bfseries #2.}]}{\end{trivlist}}
\newenvironment{problem}[2][Problem]{\begin{trivlist}
\item[\hskip \labelsep {\bfseries #1}\hskip \labelsep {\bfseries #2.}]}{\end{trivlist}}
\newenvironment{question}[2][Question]{\begin{trivlist}
\item[\hskip \labelsep {\bfseries #1}\hskip \labelsep {\bfseries #2.}]}{\end{trivlist}}
\newenvironment{corollary}[2][Corollary]{\begin{trivlist}
\item[\hskip \labelsep {\bfseries #1}\hskip \labelsep {\bfseries #2.}]}{\end{trivlist}}
 
\begin{document}
 
\title{Homework 2}
\author{Ziming Ji\\ 
PHY 539: Introduction to String Theory}
 
\maketitle
 
\section{Problem 1}
First we find the mode expansion of the space-time Lorentz generator $J^{\mu\nu}$ of a bosonic open string. By definition we have 
\begin{equation}
J^{\mu\nu}=\int_{0}^{\pi} J^{\mu\nu}_0 d\sigma = \frac{1}{\pi l_s^2}\int_{0}^{\pi}(X^\mu \dot{X}^\nu-X^\nu \dot{X}^\mu) d\sigma.
\end{equation}
Plugging in the mode expansion of $X$ field and its derivative:
\begin{equation}
\begin{aligned}
X^\mu(\tau,\sigma)=x^\mu+l_s^2 p^\mu \tau+il_s\sum\limits_{m\neq 0}&\frac{1}{m}\alpha^\mu_m e^{-im\tau}\text{cos}(m\sigma)\\
\dot{X}^\mu(\tau,\sigma)=l_s^2 p^\mu+l_s\sum\limits_{m\neq 0}\alpha^\mu_m &e^{-im\tau}\text{cos}(m\sigma),
\end{aligned}
\end{equation}
we obtain(omitting terms that vanish after integration):
\begin{equation}
\begin{aligned}
J^{\mu\nu}=&\frac{1}{\pi l_s^2}\int_{0}^{\pi} d\sigma (l_s^2 x^\mu p^\nu+l_s^4 p^\mu p^\nu \tau+il_s^2\text{cos}(m\sigma)^2\sum\limits_{m=1}^\infty\frac{1}{m}(\alpha_m^\mu \alpha_{-m}^\nu-\alpha_{-m}^\mu \alpha_m^\nu+\alpha_m^\mu \alpha_{m}^\nu e^{-2im\tau}-\alpha_{-m}^\mu \alpha_{-m}^\nu e^{2im\tau})\\
&-l_s^2 x^\nu p^\mu-l_s^4 p^\nu p^\mu \tau-il_s^2\text{cos}(m\sigma)^2\sum\limits_{m=1}^\infty\frac{1}{m}(\alpha_m^\nu \alpha_{-m}^\mu-\alpha_{-m}^\nu \alpha_m^\mu+\alpha_m^\nu \alpha_{m}^\mu e^{-2im\tau}-\alpha_{-m}^\nu \alpha_{-m}^\mu e^{2im\tau}))\\
=&\frac{1}{\pi l_s^2}[\pi l_s^2(x^\mu p^\nu-x^\nu p^\mu)+il_s^2 \frac{\pi}{2}\sum\limits_{m=1}^\infty\frac{1}{m}(2\alpha_m^\mu \alpha_{-m}^\nu-2\alpha_m^\nu \alpha_{-m}^\mu)]\\
=&x^\mu p^\nu-x^\nu p^\mu+i\sum\limits_{m=1}^\infty\frac{1}{m}(\alpha_m^\mu \alpha_{-m}^\nu-\alpha_m^\nu \alpha_{-m}^\mu).
\end{aligned}
\end{equation}
Applying the canonical commutation relation:
\begin{equation}
[x^\mu, p^\nu]=i\eta^{\mu\nu},\quad[x^\mu, x^\nu]=[p^\mu, p^\nu]=0,\quad[\alpha^\mu_m, \alpha^\nu_n]=m\eta^{\mu\nu}\delta_{m+n,0},
\end{equation}
we read that $[p^\mu, p^\nu]=0$ immediately. And it is not hard to see that
\begin{equation}
[p^\mu, J^{\nu\sigma}]=[p^\mu, x^\nu p^\sigma-x^\sigma p^\nu]=i\eta^{\sigma\mu}p^\nu-i\eta^{\nu\mu}p^\sigma.
\end{equation}
Lastly, $[J^{\mu\nu}, J^{\sigma\lambda}]$ can be separated into two terms:
\begin{equation}
\begin{aligned}[t]
&[J^{\mu\nu}, J^{\sigma\lambda}]=[x^\mu p^\nu-x^\nu p^\mu, x^\sigma p^\lambda-x^\lambda p^\sigma]\\
&-[\sum\limits_{m=1}^\infty\frac{1}{m}(\alpha_m^\mu \alpha_{-m}^\nu-\alpha_m^\nu \alpha_{-m}^\mu),\sum\limits_{m=1}^\infty\frac{1}{m}(\alpha_m^\sigma \alpha_{-m}^\lambda-\alpha_m^\lambda \alpha_{-m}^\sigma)]
\end{aligned}
\end{equation}
First we have $[x^\mu p^\nu,x^\sigma p^\lambda]=i\eta^{\mu\lambda}x^\sigma p^\nu-i\eta^{\sigma\nu}x^\mu p^\lambda$. Applying this we get:
\begin{equation} \label{commu1}
\begin{aligned}[t]
[x^\mu p^\nu-x^\nu p^\mu &, x^\sigma p^\lambda-x^\sigma p^\lambda]=i\eta^{\mu\lambda}x^\sigma p^\nu-i\eta^{\sigma\nu}x^\mu p^\lambda-i\eta^{\mu\sigma}x^\lambda p^\nu+i\eta^{\lambda\nu}x^\mu p^\sigma\\
&-i\eta^{\nu\lambda}x^\sigma p^\mu+i\eta^{\sigma\mu}x^\nu p^\lambda+i\eta^{\nu\sigma}x^\lambda p^\mu-i\eta^{\lambda\mu}x^\nu p^\sigma.
\end{aligned}
\end{equation}
Second, after applying $[\alpha_m^\mu \alpha_{-m}^\nu, \alpha_m^\sigma \alpha_{-m}^\lambda]=m\eta^{\mu\lambda}a_m^\sigma a_{-m}^\nu-m\eta^{\nu\sigma}a_m^\mu a_{-m}^\lambda$, we have
\begin{equation} \label{commu2}
\begin{aligned}[t]
&[\sum\limits_{m=1}^\infty\frac{1}{m}(\alpha_m^\mu \alpha_{-m}^\nu-\alpha_m^\nu \alpha_{-m}^\mu),\sum\limits_{m=1}^\infty\frac{1}{m}(\alpha_m^\sigma \alpha_{-m}^\lambda-\alpha_m^\lambda \alpha_{-m}^\sigma)]=\sum\limits_{m=1}^\infty\frac{1}{m}(\eta^{\mu\lambda}a_m^\sigma a_{-m}^\nu-\eta^{\nu\sigma}a_m^\mu a_{-m}^\lambda\\
&-\eta^{\mu\sigma}a_m^\lambda a_{-m}^\nu+\eta^{\nu\lambda}a_m^\mu a_{-m}^\sigma-\eta^{\nu\lambda}a_m^\sigma a_{-m}^\mu+\eta^{\mu\sigma}a_m^\nu a_{-m}^\lambda+\eta^{\nu\sigma}a_m^\lambda a_{-m}^\mu-\eta^{\mu\lambda}a_m^\nu a_{-m}^\sigma).
\end{aligned}
\end{equation}
It is not hard to see from now that matching equation \ref{commu1} and \ref{commu2}, one can obtain the desired commutation relation for Lorentz generators:
\begin{equation}
\begin{aligned}[t]
[J^{\mu\nu}, J^{\sigma\lambda}]=-i\eta^{\nu\sigma}J^{\mu\lambda}+i\eta^{\mu\sigma}J^{\nu\lambda}+i\eta^{\nu\lambda}J^{\mu\sigma}-i\eta^{\mu\lambda}J^{\nu\sigma}.
\end{aligned}
\end{equation}
\section{Problem 2}
\begin{paragraph}{a)}
First massive level of $D=26$ open string is the combination of two kinds of excitations: $\alpha_{-1}^\mu\alpha_{-1}^\nu\ket{0;k}$ and $\alpha_{-2}\ket{0;k}$. A general state is $(s_{\mu\nu}\alpha_{-1}^\mu\alpha_{-1}^\nu+v_\mu\alpha_{-2}^\mu)\ket{0;k}$. Physical state conditions are the constraints that $L_m\ket{\phi}=0$ for $m>0$ and $(L_0-1)\ket{\phi}=0$.  The second one, also called mass shell condition, gives $\alpha' M^2=1$ or $-\alpha' k^\mu k_\mu=1$. The first one need to be examined more carefully. It is not hard to see that only $L_1 \text{and } L_2$ conditions are not manifestly zero:
\begin{equation}
\begin{aligned}[t]
L_1\alpha_{-2}^\mu\ket{0;k}&=\frac{1}{2}\sum\limits_{n=-\infty}^\infty\alpha_{1-n}^\lambda\alpha_n^\sigma\eta_{\lambda\sigma}\alpha_{-2}^\mu\ket{0;k}=\frac{1}{2}(\alpha_{2}^\lambda\alpha_{-1}^\sigma\eta_{\lambda\sigma}\alpha_{-2}^\mu+\alpha_{-1}^\lambda\alpha_{2}^\sigma\eta_{\lambda\sigma}\alpha_{-2}^\mu)\ket{0;k}\\
&=(\alpha_{-1}^\sigma\eta_{\lambda\sigma}\eta^{\lambda\mu}+\alpha_{-1}^\lambda\eta_{\lambda\sigma}\eta^{\sigma\mu})\ket{0;k}=2\alpha_{-1}^{\mu}\ket{0;k},\\
L_1\alpha_{-1}^\mu\alpha_{-1}^\nu\ket{0;k}&=\frac{1}{2}\sum\limits_{n=-\infty}^\infty\alpha_{1-n}^\lambda\alpha_n^\sigma\eta_{\lambda\sigma}\alpha_{-1}^\mu\alpha_{-1}^\nu\ket{0;k}=\frac{1}{2}(\alpha_{1}^\lambda\alpha_0^\sigma\eta_{\lambda\sigma}\alpha_{-1}^\mu\alpha_{-1}^\nu+\alpha_{0}^\lambda\alpha_1^\sigma\eta_{\lambda\sigma}\alpha_{-1}^\mu\alpha_{-1}^\nu)\ket{0;k}\\
&=\frac{1}{2}[\frac{1}{2}l_s k^\sigma\eta_{\lambda\sigma}(\eta^{\lambda\mu}\alpha_{-1}^\nu+\eta^{\lambda\nu}\alpha_{-1}^\mu)+\frac{1}{2}l_s k^\lambda\eta_{\lambda\sigma}(\eta^{\sigma\mu}\alpha_{-1}^{\nu}+\eta^{\sigma\nu}\alpha_{-1}^{\mu})]\ket{0;k}\\
&=\frac{1}{2}l_s(k^\mu\alpha_{-1}^\nu+k^\nu\alpha_{-1}^\mu)\ket{0;k}.
\end{aligned}
\end{equation}
Thus $L_1\ket{\phi}$ means $\frac{1}{2}l_s s_{\mu\nu}k^\mu+v_\nu=0$. \\

Similarly, when $m=2$, $L_2\ket{0;k}=0$ gives: $s_{\mu}^\mu+l_s v_\sigma k^\sigma=0$. We have $\frac{D^2+D}{2}+D-D-1=350$ degrees of freedom right now. We will reduce them after considering gauge invariance.
\end{paragraph}
\begin{paragraph}{b)}
Null state meets physical conditions but is orthogonal to every physical state including itself. These are states with zero norm: $\bra{\phi}\ket{\phi}=0$: (real coefficients?)
\begin{equation}
\begin{aligned}[t]
\bra{0;k}(s_{\delta\gamma}\alpha_{1}^\gamma\alpha_{1}^\delta+v_\lambda\alpha_{2}^\lambda)(s_{\mu\nu}\alpha_{-1}^\mu\alpha_{-1}^\nu+v_\sigma\alpha_{-2}^\sigma)\ket{0;k}=0\\
\bra{0;k}(s_{\delta\gamma} s_{\mu\nu} \alpha_1^\gamma\alpha_1^\delta\alpha_{-1}^\mu\alpha_{-1}^\nu+v_\lambda v_\sigma \alpha_2^\lambda \alpha_{-2}^\sigma)\ket{0;k}=0\\
\bra{0;k}[s_{\delta\gamma} s_{\mu\nu} (\eta^{\gamma\mu}\eta^{\delta\nu}+\eta^{\gamma\nu}\eta^{\mu\delta}) +2 v_\lambda v_\sigma \eta^{\lambda\sigma}]\ket{0;k}=0\\
\bra{0;k}[s_{\delta\gamma} s_{\mu\nu} (\eta^{\gamma\mu}\eta^{\delta\nu}+\eta^{\gamma\nu}\eta^{\mu\delta}) +2 v_\lambda v_\sigma \eta^{\lambda\sigma}]\ket{0;k}=0.
\end{aligned}
\end{equation}
This tells us that if coefficients have relation $s_\delta^\mu s_\mu^\delta+v_\lambda v^\lambda=0$, the state is null. And $\bra{\phi}\ket{\psi}=0$ for an arbitrary physical state $\ket{\psi}$ tells us that $s_\delta^\mu s_\mu'^\delta+v_\lambda v'^{\lambda}=0$.\\

This seems hopeless. One should instead use the construction of null states basis and one can find $D+1-1=D$ independent basis($L_{-1}c_\mu\alpha_{-1}^\mu\ket{0;k}$ and $(L_{-2}+\frac{3}{2}L_{-1}^2)\ket{0;k}$ with $c_\mu k^\mu=0$). A gauge transformation is a shift like $\ket{\phi}+A_\mu\ket{null}^\mu$. This will not change any measurement $\bra{\phi}\mathcal{O}\ket{\phi}$.
\end{paragraph}
\begin{paragraph}{c)}
Thus the total dof is $350-26=324$. The rest is to show that the states transfer under $SO(25)$. ?
\end{paragraph}
\section{Problem 3}
A general state is $t_{\beta\theta\xi}\alpha_{-1}^\beta\alpha_{-1}^\theta\alpha_{-1}^\xi+s_{\mu\nu}\alpha_{-2}^\mu\alpha_{-1}^\nu+v_\sigma\alpha_{-3}^\sigma\ket{0;k}$. $s_{\mu\nu}$ is not by definition symmetric any more. Mass-shell condition gives $-\alpha' k^\mu k_\mu=2$. $L_m\ket{\phi}=0$ gives non-trivial conditions only for $m=1,2,\text{ and }3$.
\begin{equation}
\begin{aligned}[t]
L_1\ket{\phi}=\frac{1}{2}\sum\limits_{n=-\infty}^\infty \alpha_{1-n}^\gamma\alpha_n^\delta\eta_{\gamma\delta}(t_{\beta\theta\xi}\alpha_{-1}^\beta\alpha_{-1}^\theta\alpha_{-1}^\xi+s_{\mu\nu}\alpha_{-2}^\mu\alpha_{-1}^\nu+v_\sigma\alpha_{-3}^\sigma\ket{0;k})=0\\
\frac{1}{2}[2\alpha_{0}^\gamma\alpha_1^\delta\eta_{\gamma\delta}t_{\beta\theta\xi}\alpha_{-1}^\beta\alpha_{-1}^\theta\alpha_{-1}^\xi+(2\alpha_{0}^\gamma\alpha_1^\delta\eta_{\gamma\delta}+\alpha_{-1}^\gamma\alpha_2^\delta\eta_{\gamma\delta})s_{\mu\nu}\alpha_{-2}^\mu\alpha_{-1}^\nu+\alpha_{-2}^\gamma\alpha_3^\delta\eta_{\gamma\delta}v_\sigma\alpha_{-3}^\sigma]\ket{0;k}=0\\
\frac{1}{2}[2\alpha_{0}^\gamma\eta_{\gamma\delta}t_{\beta\theta\xi}(\eta^{\delta\beta}\alpha_{-1}^\theta\alpha_{-1}^\xi+\eta^{\theta\delta}\alpha_{-1}^\beta\alpha_{-1}^\xi+\eta^{\delta\xi}\alpha_{-1}^\beta\alpha_{-1}^\theta)+s_{\mu\nu}(2\alpha_{0}^\gamma\alpha_{-2}^\mu\eta^{\delta\nu}\eta_{\gamma\delta}+2\alpha_{-1}^\gamma\alpha_{-1}^\nu\eta^{\delta\mu}\eta_{\gamma\delta})\\
+3\alpha_{-2}^\gamma\eta_{\gamma\delta}v_\sigma\eta^{\delta\sigma}]\ket{0;k}=0\\
\frac{1}{2}[2t_{\beta\theta\xi}(\alpha_{0}^\beta\alpha_{-1}^\theta\alpha_{-1}^\xi+\alpha_{0}^\theta\alpha_{-1}^\beta\alpha_{-1}^\xi+\alpha_{0}^\xi\alpha_{-1}^\beta\alpha_{-1}^\theta)+s_{\mu\nu}(2\alpha_{0}^\nu\alpha_{-2}^\mu+2\alpha_{-1}^\mu\alpha_{-1}^\nu)+3v_\sigma\alpha_{-2}^\sigma]\ket{0;k}=0
\end{aligned}
\end{equation}
The constraints that come from $L_1$ are $l_s s_{\mu\nu}k^{\nu}+3v_\mu=0$ and $3l_s t_{\beta\theta\xi}k^\beta+2s_{\theta\xi}=0$.
\begin{equation}
\begin{aligned}[t]
L_2\ket{\phi}=\frac{1}{2}\sum\limits_{n=-\infty}^\infty \alpha_{2-n}^\gamma\alpha_n^\delta\eta_{\gamma\delta}(t_{\beta\theta\xi}\alpha_{-1}^\beta\alpha_{-1}^\theta\alpha_{-1}^\xi+s_{\mu\nu}\alpha_{-2}^\mu\alpha_{-1}^\nu+v_\sigma\alpha_{-3}^\sigma\ket{0;k})=0\\
\frac{1}{2}[\alpha_{1}^\gamma\alpha_1^\delta\eta_{\gamma\delta}t_{\beta\theta\xi}\alpha_{-1}^\beta\alpha_{-1}^\theta\alpha_{-1}^\xi+2\alpha_{0}^\gamma\alpha_2^\delta\eta_{\gamma\delta}s_{\mu\nu}\alpha_{-2}^\mu\alpha_{-1}^\nu+\alpha_{-1}^\gamma\alpha_3^\delta\eta_{\gamma\delta}v_\sigma\alpha_{-3}^\sigma]\ket{0;k}=0\\
\frac{1}{2}[6t_{\beta\theta\xi}\eta^{\theta\xi}\alpha_{-1}^\beta+2l_s k^\mu s_{\mu\nu}\alpha_{-1}^\nu+3v_\sigma\alpha_{-1}^\sigma]\ket{0;k}=0
\end{aligned}
\end{equation}
The constraint that comes from $L_2$ is $6t_{\beta\theta\xi}\eta^{\theta\xi}+2l_s k^\mu s_{\mu\beta}+3v_\beta=0$.
\begin{equation}
\begin{aligned}[t]
L_3\ket{\phi}=\frac{1}{2}\sum\limits_{n=-\infty}^\infty \alpha_{3-n}^\gamma\alpha_n^\delta\eta_{\gamma\delta}(t_{\beta\theta\xi}\alpha_{-1}^\beta\alpha_{-1}^\theta\alpha_{-1}^\xi+s_{\mu\nu}\alpha_{-2}^\mu\alpha_{-1}^\nu+v_\sigma\alpha_{-3}^\sigma\ket{0;k})=0\\
\frac{1}{2}[2\alpha_{2}^\gamma\alpha_1^\delta\eta_{\gamma\delta}s_{\mu\nu}\alpha_{-2}^\mu\alpha_{-1}^\nu+2\alpha_{0}^\gamma\alpha_3^\delta\eta_{\gamma\delta}v_\sigma\alpha_{-3}^\sigma]\ket{0;k}=\frac{1}{2}[4s_{\mu}^\mu+3l_s k^\sigma v_\sigma]\ket{0;k}=0
\end{aligned}
\end{equation}
The constraint that comes from $L_3$ is $4s_{\mu}^\mu+3l_s k^\sigma v_\sigma=0$.\\

Now we have $\frac{D(D-1)(D-2)}{6}+3D(D-1)+D+\frac{D^2+D}{2}+D-D-\frac{D^2+D}{2}-D-1=4549$ degrees of freedom. Gauge invariance will reduce it further.
\end{document}
